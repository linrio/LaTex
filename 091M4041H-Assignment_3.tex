\documentclass[11pt]{article}
\usepackage{CJK}
\usepackage[top=2cm, bottom=2cm, left=2cm, right=2cm]{geometry}
\usepackage{algorithm}
\usepackage{algorithmicx}
\usepackage{algpseudocode}
\usepackage{amsmath}
\usepackage{enumerate}
\usepackage{tikz}

%\floatname{algorithm}{算法}
%\renewcommand{\algorithmicrequire}{\textbf{输入:}}
%\renewcommand{\algorithmicensure}{\textbf{输出:}}
    \title{091M4041H - Assignment 3}
    \author{No. 201628017729008,  Lin Lingfeng}

\begin{document}
\begin{CJK*}{GBK}{song}
    \maketitle

    \section{whether there exists an undirected graph}

    \paragraph{} Given a list of n natural numbers $d_1, d_2,...,d_n$, show how to decide in polynomial time whether there exists an undirected graph $G = (V, E)$ whose node degrees are precisely the numbers $d_1, d_2,...,d_n$. $G$ should not contain multiple edges between the same pair of nodes, or "loop" edges with both endpoints equal to the same node.

    \paragraph{}\textbf{Solution:}\\
    题目的意思是:给定$n$ 个自然数 $d_1, d_2,...,d_n$,使得构成的无向图。该无向图的每个顶点的度正好是$d_1, d_2,...,d_n$。易知,当某个$d_i$的值大于n时,无法构成无向图。因此首先判断是否大于$n$。即对$d_1, d_2,...,d_n$进行从大到小排序。假设得到的$d_1, d_2,...,d_n$是排好序的。当$d_1$ 小于$n$时,对$d_2, d_3...$直到第$d_1$个后面的节点依次减一。代表$d_1$与其紧跟其后的节点相连。如果出现小于零的情况,则代表节点不够$d_1$连了,所以无法构成无向图。如果没有出现小于零,则$n-1$对后续节点重复$d_1$节点的操作。

%    $$OPT(s_i, s_j)=
%    \begin{cases}
%        POT(s_{i+1}, s_j)&                             if S_i\%S_j != 0 or S_j\%S_i != 0 \\
%        POT(s_i, s_{j+1})&                             if S_i\%S_j = 0 or S_j\%S_i = 0
%    \end{cases}$$

    \paragraph{}\textbf{Algorithm:}\\
    \begin{algorithm}
        \caption{whether there exists an undirected graph}
        \begin{algorithmic}[1]
            \Require $dataset$: $D = d_1, d_2,...,d_n$, \qquad The size of $D$: $n$
            \Ensure ture or false
            \Function {OPT}{$i,j$}
                \While{n}
                    \State sort(D)
                    \State $d_1$ = d[0]
                    \If {$d_1 > n$}
                        \State throw error
                    \Else
                        \For {$i=1$ to $d_1$}
                            d[i] -= 1
                            \If{d[i] $<$ 0}
                                \State throw error
                            \EndIf
                        \EndFor
                        \State d[0] = 0
                        \State n -= 1
                    \EndIf
                \EndWhile
                \State return ture
            \EndFunction
        \end{algorithmic}
    \end{algorithm}



The \textbf{Lemma} is
    \begin{equation*}
    G(\{d_1, d_2,...,d_n\}) \leftrightarrow G(\{d_2 - 1, d_3 - 1,...,d_{d_1+1} - 1, d_{d_1+2},...,d_n\})
    \end{equation*}

\paragraph{}\textbf{Correctness Proof:}
取最大度数的点是为了后面有足够多的点可以抵消该点的度数,首先随机找出一个节点,连接其他$d_i$个节点,之后就不再考虑该节点,然后在剩下的节点中找度数最大的节点,给它分配第二大的度数,即与$d_{i+1}$个节点连接,如此类推,直到分配完所有度数。


\paragraph{}\textbf{Runtime Analysis:} 一个遍历中嵌套一个排序,所以算法的时间复杂度为$O(n^2 logn)$.

\section{Partition}

%\paragraph{}There are n distinct jobs, labeled $J_1, J_2,..., J_n$ , which can be performed completely independently of one another. Each job consists of two stages: ?rst it needs to be preprocessed on the supercomputer, and then it needs to be finished on one of the PCs. Let`s say that job $J_i$ needs $p_i$ seconds of time on the supercomputer, followed by $f_i$ seconds of time on a PC. Since there are at least $n$ PCs available on the premises, the ?nishing of the jobs can be performed on PCs at the same time. However, the supercomputer can only work on a single job a time without any interruption. For every job, as soon as the preprocessing is done on the supercomputer, it can be handed off to a PC for finishing.

\paragraph{}Suppose you are given two sets $A$ and $B$, each containing n positive integers. You can choose to reorder each set however you like. After reordering, let $a_i$ be the $ith$ element of set $A$, and let $b_i$ be the ith element of set B. You then receive a payoff of $\prod^{n}_{i=1} a^{b_i}_i$. Give an polynomial-time algorithm that will maximize your payoff.

\paragraph{}\textbf{Solution:}\\
因为对于指数函数,$y = \prod^{n}_{i=1} a^{b_i}_i$ 来说,$y$ 是 $a$ 的增函数,$y$也是$b$的增函数,对$y = \prod^{n}_{i=1} a^{b_i}_i$ 取对数,得到$log y = \sum^{n}_{i=1} log(a^{b_i}_i) = \sum^{n}_{i=1} b^i log(a_i)$。因为$a_i, b_i$都是正数,不存在偶函数性质。当底数 $a_i$ 的值大于自然底数 $e$ 时, $b_i log(a_i)$是以正数增加 如图 \ref{fig 1},所以$a_i, b_i$越大越好。 所以要使得 $y$ 达最大,排个序即可。

\begin{figure*}[!t]
\centering
\includegraphics[width=5in]{3.jpg}
\caption{The function of $y = x$ and $y = logx$}
\label{fig 1}
\end{figure*}

    \paragraph{}\textbf{Algorithm:}\\
    \begin{algorithm}
        \caption{maximize your payoff}
        \begin{algorithmic}[1] %每行显示行号
            \Require $database$: $A, B$, \qquad The size of $A, B$: $n$
            \Ensure payoff
            \Function {maximize your payoff}{$A,B$}
                \State sort(A)
                \State sort(B)
                \For{$i = 1$ to n}

                        \State payoff = $\prod^{n}_{i=1} a^{b_i}_{i}$

                \EndFor
                \State \Return payoff
            \EndFunction
        \end{algorithmic}
    \end{algorithm}


\paragraph{}\textbf{Correctness Proof:}\\
假设$a^{b_i}_{i}$ 是各自队列最大的2个数,用较小数$b_i > b_j $代替,则$a^{b_j}_{i}$ 与$a^{b_i}_{j}$ ,和 $a^{b_i}_{i}$ 与 $a^{b_j}_{j}$

\begin{equation*}
a^{b_j}_{i} a^{b_i}_{j} = a^{b_j}_{i} a^{b_i - b_j}_{j} a^{b_j}_{j}
                        \leq a^{b_j}_{i} a^{b_i - b_j}_{i} a^{b_j}_{j}
                        \leq a^{b_j}_{i} a^{-b_j}_{i} a^{b_i}_{i} a^{b_j}_{j}
                        = a^{b_i}_{i} a^{b_j}_{j}
\end{equation*}


\paragraph{}\textbf{Runtime Analysis:} \\
$n$次循环,2个排序。时间为:$O(n log n)$

\end{CJK*}
\end{document} 
